\begin{longtable}{|>{\raggedright}p{0.23\textwidth}|>{\raggedright}p{0.08\textwidth}|>{\raggedright}p{0.05\textwidth}|>{\raggedright}p{0.41\textwidth}|>{\raggedright}p{0.1\textwidth}|}
\hline
\hline 
\textbf{\footnotesize{Parameter }} & \textbf{\footnotesize{Data Type}} & \textbf{\footnotesize{Req?}} & \textbf{\footnotesize{Brief description}} & \textbf{\footnotesize{Location}}\tabularnewline
\hline 
\hline
\endfirsthead
\hline
\hline 
\textbf{\footnotesize{Parameter }} & \textbf{\footnotesize{Data Type}} & \textbf{\footnotesize{Req?}} & \textbf{\footnotesize{Brief description}} & \textbf{\footnotesize{Location}}\tabularnewline
\hline 
\hline
\endhead
\hline 
{\footnotesize{accept}} & {\footnotesize{String}} & {\footnotesize{No}} & {\footnotesize{Specifies the format of the body of the response. Valid
values are: }}{\footnotesize \par}
\begin{itemize}
\item {\footnotesize{application/json}}{\footnotesize \par}
\item {\footnotesize{application/xml}}{\footnotesize \par}
\end{itemize}
{\footnotesize{The default value is application/json. Per rfc2616:
\textquotedbl{}If no Accept header field is present, then it is assumed
that the client accepts all media types.\textquotedbl{} By default
our services return application/json.}}{\footnotesize \par}

{\footnotesize{Normal Accept header processing rules shall be followed
according to rfc2616.}}{\footnotesize \par}

\emph{\footnotesize{Note}}{\footnotesize{: If there is no entity body
in a normal successful response, this parameter can still be specified
to determine the format in the case of an error response message.}} & {\footnotesize{Header}}\tabularnewline
\hline 
\end{longtable}
