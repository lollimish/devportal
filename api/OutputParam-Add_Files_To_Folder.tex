{\footnotesize{}}%
\begin{longtable}{|>{\raggedright}p{0.23\textwidth}|>{\raggedright}p{0.08\textwidth}|>{\raggedright}p{0.05\textwidth}|>{\raggedright}p{0.41\textwidth}|>{\raggedright}p{0.1\textwidth}|}
\hline
\hline 
\textbf{\footnotesize{Parameter }} & \textbf{\footnotesize{Data Type}} & \textbf{\footnotesize{Req?}} & \textbf{\footnotesize{Brief description}} & \textbf{\footnotesize{Location}}\tabularnewline
\hline 
\hline
\endfirsthead
\hline
\hline 
\textbf{\footnotesize{Parameter }} & \textbf{\footnotesize{Data Type}} & \textbf{\footnotesize{Req?}} & \textbf{\footnotesize{Brief description}} & \textbf{\footnotesize{Location}}\tabularnewline
\hline 
\hline
\endhead
\hline 
{\footnotesize{content-type}} & {\footnotesize{string}} & {\footnotesize{yes}} & {\footnotesize{Specifies the type of content of the body of the entity.
Must be set to one of the following values:}}{\footnotesize \par}
\begin{itemize}
\item {\footnotesize{application/json}}{\footnotesize \par}
\item {\footnotesize{aplication/xml}}{\footnotesize \par}
\end{itemize}
{\footnotesize{Note: If there is no entity body in a normal successful
response, this parameter is still needed to specify the format in
the case of an error response message.}} & {\footnotesize{Header}}\tabularnewline
\hline 
\end{longtable}{\footnotesize \par}

{\footnotesize{}}
