\begin{longtable}{|>{\raggedright}p{0.23\textwidth}|>{\raggedright}p{0.08\textwidth}|>{\raggedright}p{0.05\textwidth}|>{\raggedright}p{0.41\textwidth}|>{\raggedright}p{0.1\textwidth}|}
\hline
\hline 
\textbf{\footnotesize{Parameter }} & \textbf{\footnotesize{Data Type}} & \textbf{\footnotesize{Req?}} & \textbf{\footnotesize{Brief description}} & \textbf{\footnotesize{Location}}\tabularnewline
\hline 
\hline
\endfirsthead
\hline
\hline 
\textbf{\footnotesize{Parameter }} & \textbf{\footnotesize{Data Type}} & \textbf{\footnotesize{Req?}} & \textbf{\footnotesize{Brief description}} & \textbf{\footnotesize{Location}}\tabularnewline
\hline 
\hline
\endhead
\hline 
{\footnotesize{accept}} & {\footnotesize{String}} & {\footnotesize{No}} & {\footnotesize{Specifies the format of the body of the response. Valid
values are: }}{\footnotesize \par}
\begin{itemize}
\item {\footnotesize{application/json}}{\footnotesize \par}
\item {\footnotesize{application/xml}}{\footnotesize \par}
\end{itemize}
{\footnotesize{The default value is application/json. Per rfc2616:
\textquotedbl{}If no Accept header field is present, then it is assumed
that the client accepts all track types.\textquotedbl{} By default
our services return application/json.}}{\footnotesize \par}

{\footnotesize{Normal Accept header processing rules shall be followed
according to rfc2616.}}{\footnotesize \par}

\emph{\footnotesize{Note}}{\footnotesize{: If there is no entity body
in a normal successful response, this parameter can still be specified
to determine the format in the case of an error response message.}} & {\footnotesize{Header}}\tabularnewline
\hline 
{\footnotesize{accept-encoding}} & {\footnotesize{String}} & {\footnotesize{No }} & {\footnotesize{Specifies the accept encoding format. When this header
is present, the gateway should compress the response to indicate the
client that response is in compressed format. Valid values are: gzip}} & {\footnotesize{Header}}\tabularnewline
\hline 
{\footnotesize{authorization}} & {\footnotesize{String}} & {\footnotesize{Yes}} & {\footnotesize{Specifies the authorization type and token. \textquotedbl{}Bearer\textquotedbl{}
+ OAuth Token – access\_token. If the authorization header is missing,
the system shall return an HTTP 400 Invalid Request message. If the
token is invalid the system shall return an HTTP 401 Unauthorized
message with a WWW-Authenticate HTTP header.}} & {\footnotesize{Header}}\tabularnewline
\hline 
{\footnotesize{content-length}} & {\footnotesize{Integer}} & {\footnotesize{No}} & {\footnotesize{Specifies the length of the content in octets. This
header parameter is only required for the non-streaming request.}} & {\footnotesize{Header}}\tabularnewline
\hline 
{\footnotesize{content-type}} & {\footnotesize{String}} & {\footnotesize{Yes}} & {\footnotesize{Specifies the type of content of the body of the entity.
Must be set to one of the following values:}}{\footnotesize \par}
\begin{itemize}
\item {\footnotesize{application/json}}{\footnotesize \par}
\item {\footnotesize{application/xml}}\end{itemize}
 & {\footnotesize{Header}}\tabularnewline
\hline 
{\footnotesize{trackId}} & {\footnotesize{Integer}} & {\footnotesize{Yes}} & {\footnotesize{Specifies the id of the track that needs to be updated}} & {\footnotesize{URI Path}}\tabularnewline
\hline 
{\footnotesize{track}} & {\footnotesize{Object}} & {\footnotesize{Yes}} & {\footnotesize{Contains the response information.}} & {\footnotesize{Body}}\tabularnewline
\hline 
{\footnotesize{name}} & {\footnotesize{String}} & {\footnotesize{No }} & {\footnotesize{Specifies the name of the track }} & {\footnotesize{Body}}\tabularnewline
\hline 
{\footnotesize{description}} & {\footnotesize{String}} & {\footnotesize{No }} & {\footnotesize{Specifies the description of the track}} & {\footnotesize{Body}}\tabularnewline
\hline 
\end{longtable}

\begin{comment}
Need to check if the whole Content.File object is valid here!!
\end{comment}


{\footnotesize{}}
