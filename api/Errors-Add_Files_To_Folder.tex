
\subsubsection{HTTP Response Codes}

\begin{comment}
List expected response codes (indicate any variation that exists across
different versions):
\end{comment}


\begin{longtable}{|>{\raggedright}p{0.15\textwidth}|>{\raggedright}p{0.15\textwidth}|>{\raggedright}p{0.56\textwidth}|}
\hline
\hline 
\textbf{\footnotesize{Code}} & \textbf{\footnotesize{Reason Phrase}} & \textbf{\footnotesize{Description}}\tabularnewline
\hline
\endfirsthead
\hline
\hline 
\textbf{\footnotesize{Code}} & \textbf{\footnotesize{Reason Phrase}} & \textbf{\footnotesize{Description}}\tabularnewline
\hline
\endhead
\hline 
{\footnotesize{204}} & {\footnotesize{No Content}} & {\footnotesize{Successful response and the response does not include
an entity.}}\tabularnewline
\hline 
{\footnotesize{400}} & {\footnotesize{Bad Request}} & {\footnotesize{Many possible reasons not specified by the other codes.}}\tabularnewline
\hline 
{\footnotesize{401}} & {\footnotesize{Authentication Error}} & {\footnotesize{Authentication failed or was not provided.}}\tabularnewline
\hline 
{\footnotesize{403}} & {\footnotesize{Forbidden}} & {\footnotesize{Access permission error.}}\tabularnewline
\hline 
{\footnotesize{404}} & {\footnotesize{Not Found}} & {\footnotesize{The server has not found anything matching the Request-URI.
No indication is given of whether the condition is temporary or permanent.}}\tabularnewline
\hline 
{\footnotesize{405}} & {\footnotesize{Method Not Allowed}} & {\footnotesize{A request was made of a resource using a request method
not supported by that resource (e.g., using PUT on a REST resource
that only supports POST).}}\tabularnewline
\hline 
{\footnotesize{408}} & {\footnotesize{Request Timeout}} & {\footnotesize{The client did not produce a request within the time
that the server was prepared to wait. The client MAY repeat the request
without modifications at any later time. }}\tabularnewline
\hline 
{\footnotesize{411}} & {\footnotesize{Length Required}} & {\footnotesize{The Content-Length header was not specified.}}\tabularnewline
\hline 
{\footnotesize{413}} & {\footnotesize{Request Entity Too Large}} & {\footnotesize{The size of the request body exceed the maximum size
permitted.}}\tabularnewline
\hline 
{\footnotesize{415}} & {\footnotesize{Unsupported Media Type}} & {\footnotesize{The request is in a format not supported by the requested
resource for the requested method.}}\tabularnewline
\hline 
{\footnotesize{500}} & {\footnotesize{Internal Server Error}} & {\footnotesize{The server encountered an internal error or timed out;
please retry.}}\tabularnewline
\hline 
{\footnotesize{503}} & {\footnotesize{Service Unavailable}} & {\footnotesize{The server is currently unable to receive requests;
please retry.}}\tabularnewline
\hline 
\end{longtable}




\subsubsection{Service Exceptions}

\begin{comment}
List the service exceptions generated by the operation (indicate any
variation that exists across different versions):
\end{comment}


\begin{longtable}{|>{\raggedright}p{0.1\textwidth}|>{\raggedright}p{0.33\textwidth}|>{\raggedright}p{0.33\textwidth}|>{\raggedright}p{0.1\textwidth}|}
\hline
\hline 
\textbf{\footnotesize{MessageId}} & \textbf{\footnotesize{Text}} & \textbf{\footnotesize{Variables}} & \textbf{\footnotesize{Parent HTTP Code}}\tabularnewline
\hline 
\hline
\endfirsthead
\hline
\hline 
\textbf{\footnotesize{MessageId}} & \textbf{\footnotesize{Text}} & \textbf{\footnotesize{Variables}} & \textbf{\footnotesize{Parent HTTP Code}}\tabularnewline
\hline 
\hline
\endhead
\hline 
{\footnotesize{SVC0001}} & {\footnotesize{A service error has occurred. Error code is <Error
Explanation>}} & {\footnotesize{Error Explanation : <content here>}} & {\footnotesize{400}}\tabularnewline
\hline 
{\footnotesize{SVC0002}} & {\footnotesize{Invalid input value for Message part <part name>}} & {\footnotesize{part name : name of the input parameter that resulted
in the error}} & {\footnotesize{400}}\tabularnewline
\hline 
\end{longtable}




\subsubsection{Policy Exceptions}

\begin{comment}
List the policy exceptions generated by the operation (indicate any
variation that exists across different versions):
\end{comment}


\begin{longtable}{|>{\raggedright}p{0.1\textwidth}|>{\raggedright}p{0.33\textwidth}|>{\raggedright}p{0.33\textwidth}|>{\raggedright}p{0.1\textwidth}|}
\hline
\hline 
\textbf{\footnotesize{MessageId}} & \textbf{\footnotesize{Text}} & \textbf{\footnotesize{Variables}} & \textbf{\footnotesize{Parent HTTP Code}}\tabularnewline
\hline 
\hline
\endfirsthead
\hline
\hline 
\textbf{\footnotesize{MessageId}} & \textbf{\footnotesize{Text}} & \textbf{\footnotesize{Variables}} & \textbf{\footnotesize{Parent HTTP Code}}\tabularnewline
\hline 
\hline
\endhead
\hline 
{\footnotesize{POL0001}} & {\footnotesize{A policy error occurred. For example, rate limit error,
authentication and authorization error.}} & {\footnotesize{N/A}} & {\footnotesize{401,403}}\tabularnewline
\hline 
{\footnotesize{POL1009}} & {\footnotesize{User has not been provisioned for \%1}} & {\footnotesize{System that hasn’t been provisioned}} & {\footnotesize{403}}\tabularnewline
\hline 
\end{longtable}
