%% LyX 2.0.6 created this file.  For more info, see http://www.lyx.org/.
%% Do not edit unless you really know what you are doing.
\documentclass[12pt,english,usenames,dvipsnames]{article}
\usepackage{amsthm}
\usepackage{amsmath}
\usepackage{fontspec}
\setmainfont[Mapping=tex-text]{Arial}
\setsansfont[Mapping=tex-text]{Arial}
\setmonofont{Arial Narrow}
\usepackage{listings}
\lstset{backgroundcolor={\color{shadebox}},
basicstyle={\small},
breaklines={true},
frame=single,
frameround=tttt,
framerule={.5pt},
morekeywords={GET,POST,PUT,DELETE},
otherkeywords={GET,POST,PUT,DELETE,HTTP/1.1, 200,OK,appication/xml},
stringstyle={\ttfamily},
tabsize=4}
\usepackage[letterpaper]{geometry}
\geometry{verbose,tmargin=1.8in,bmargin=1.25in,lmargin=1in,rmargin=1in}
\usepackage{fancyhdr}
\pagestyle{fancy}
\setcounter{secnumdepth}{5}
\setcounter{tocdepth}{5}
\setlength{\parskip}{\bigskipamount}
\setlength{\parindent}{0pt}
\usepackage{color}
\usepackage{array}
\usepackage{verbatim}
\usepackage{longtable}

\makeatletter

%%%%%%%%%%%%%%%%%%%%%%%%%%%%%% LyX specific LaTeX commands.
%% Because html converters don't know tabularnewline
\providecommand{\tabularnewline}{\\}

%%%%%%%%%%%%%%%%%%%%%%%%%%%%%% Textclass specific LaTeX commands.

\numberwithin{equation}{section}
\numberwithin{figure}{section}
\usepackage{enumitem}		% customizable list environments
\newlength{\lyxlabelwidth}      % auxiliary length 
\numberwithin{table}{section}

%%%%%%%%%%%%%%%%%%%%%%%%%%%%%% User specified LaTeX commands.
\usepackage[usenames,dvipsnames,svgnames,table]{xcolor}
\usepackage{fancyhdr}
\usepackage{eso-pic}
\usepackage[stamp]{draftwatermark}
\usepackage{colortbl}

\SetWatermarkText{DPS Internal}
\SetWatermarkAngle{45}
\SetWatermarkLightness{0.9}
\SetWatermarkScale{.8}

\date{}
\lhead{\small{www.dedoimedo.com}}
\rhead{\small{some text}} 
\pagestyle{fancy}

\setlength{\headheight}{0.6in}
\setlength{\headwidth}{\textwidth}
\fancyhead[L]{}% empty left
\fancyhead[R]{ % right
   \includegraphics[height=0.75in]{../common/att_logo.png}
}
\pagestyle{fancy}

\definecolor{blue}{RGB}{79,129,189}
\definecolor{shadebox}{RGB}{255,255,204}

\cfoot{\scriptsize © 2013 AT\&T Intellectual Property. All rights reserved. AT\&T and AT\&T logo are trademarks of AT\&T Intellectual Property.}

\newcommand{\attcategory}[1]{\textbf{\Large Category: \textcolor{blue}{#1}}}
\newcommand{\attservice}[1]{\textbf{\Large Service:  \textcolor{blue}{#1}}}
\newcommand{\attdocumentnumber}[1]{\textbf{\small Document Number: #1}}
\newcommand{\attrevision}[1]{\textbf{\small Revision: #1}}
\newcommand{\attrevisiondate}[1]{\textbf{\small Revision Date: #1}}
\newcommand{\attauthor}[1]{\textbf{\small Author: #1}}

\let\oldtabular=\tabular
\def\tabular{\footnotesize\oldtabular}

\makeatother

\usepackage{xunicode}
\usepackage{polyglossia}
\setdefaultlanguage{english}
\begin{document}
{\footnotesize{}}%
\begin{longtable}{|>{\raggedright}p{0.23\textwidth}|>{\raggedright}p{0.08\textwidth}|>{\raggedright}p{0.05\textwidth}|>{\raggedright}p{0.41\textwidth}|>{\raggedright}p{0.1\textwidth}|}
\hline
\hline 
\textbf{\footnotesize{Parameter }} & \textbf{\footnotesize{Data Type}} & \textbf{\footnotesize{Req?}} & \textbf{\footnotesize{Brief description}} & \textbf{\footnotesize{Location}}\tabularnewline
\hline 
\hline
\endfirsthead
\hline
\hline 
\textbf{\footnotesize{Parameter }} & \textbf{\footnotesize{Data Type}} & \textbf{\footnotesize{Req?}} & \textbf{\footnotesize{Brief description}} & \textbf{\footnotesize{Location}}\tabularnewline
\hline 
\hline
\endhead
\hline 
{\footnotesize{content-type}} & {\footnotesize{String}} & {\footnotesize{yes}} & {\footnotesize{Specifies the type of content of the body of the entity.
Must be set to one of the following values:}}{\footnotesize \par}
\begin{itemize}
\item {\footnotesize{applciation/json}}{\footnotesize \par}
\item {\footnotesize{application/xml}}\end{itemize}
 & {\footnotesize{Body}}\tabularnewline
\hline 
{\footnotesize{content-encoding}} & {\footnotesize{String}} & {\footnotesize{No }} & {\footnotesize{Specifies the type of encoding used on the data. Supported
values are: identity or gzip}} & {\footnotesize{Header}}\tabularnewline
\hline 
{\footnotesize{location}} & {\footnotesize{AnyURI}} & {\footnotesize{Yes }} & {\footnotesize{Specifies the location of the newly created resource.
This URI can be used to interact with the newly created resource.
See Get Playlist Contents for how the URI is structured.}} & {\footnotesize{Header}}\tabularnewline
\hline 
{\footnotesize{playlist }} & {\footnotesize{Object}} & {\footnotesize{Yes}} & {\footnotesize{Contains the response information for creating a playlist}} & {\footnotesize{Body}}\tabularnewline
\hline 
\end{longtable}{\footnotesize \par}


\paragraph*{Playlist Object}

This table describes the parameters that belong to the playlist Object

{\footnotesize{}}%
\begin{longtable}{|>{\raggedright}p{0.23\textwidth}|>{\raggedright}p{0.08\textwidth}|>{\raggedright}p{0.05\textwidth}|>{\raggedright}p{0.41\textwidth}|>{\raggedright}p{0.1\textwidth}|}
\hline
\hline 
\textbf{\footnotesize{Parameter }} & \textbf{\footnotesize{Data Type}} & \textbf{\footnotesize{Req?}} & \textbf{\footnotesize{Brief description}} & \textbf{\footnotesize{Location}}\tabularnewline
\hline 
\hline
\endfirsthead
\hline
\hline 
\textbf{\footnotesize{Parameter }} & \textbf{\footnotesize{Data Type}} & \textbf{\footnotesize{Req?}} & \textbf{\footnotesize{Brief description}} & \textbf{\footnotesize{Location}}\tabularnewline
\hline 
\hline
\endhead
\hline 
{\footnotesize{id}} & {\footnotesize{String}} & {\footnotesize{Yes}} & {\footnotesize{Specifies the Id of the content object}} & {\footnotesize{Body}}\tabularnewline
\hline 
{\footnotesize{name}} & {\footnotesize{String}} & {\footnotesize{No }} & {\footnotesize{Specifies the name of content object}} & {\footnotesize{Body}}\tabularnewline
\hline 
{\footnotesize{creationDate}} & {\footnotesize{String}} & {\footnotesize{No}} & {\footnotesize{Specifies the date of playlist creation in the system.}} & {\footnotesize{Body}}\tabularnewline
\hline 
{\footnotesize{size}} & {\footnotesize{Integer}} & {\footnotesize{No}} & {\footnotesize{Specifies the size of the playlist in bytes}} & {\footnotesize{Body}}\tabularnewline
\hline 
{\footnotesize{description}} & {\footnotesize{String}} & {\footnotesize{No}} & {\footnotesize{Specifies the description of the entry}} & {\footnotesize{Body}}\tabularnewline
\hline 
{\footnotesize{shareCodes}} & {\footnotesize{Array}} & {\footnotesize{No}} & {\footnotesize{Specifies an array of Sshare codes that this media
object belongs to.}} & {\footnotesize{Body}}\tabularnewline
\hline 
\end{longtable}{\footnotesize \par}



{\footnotesize{}}
\end{document}
