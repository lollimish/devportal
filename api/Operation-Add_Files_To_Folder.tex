%% LyX 2.0.6 created this file.  For more info, see http://www.lyx.org/.
%% Do not edit unless you really know what you are doing.
\documentclass[12pt,english,usenames,dvipsnames]{article}
\usepackage{amsthm}
\usepackage{amsmath}
\usepackage{fontspec}
\setmainfont[Mapping=tex-text]{Arial}
\setsansfont[Mapping=tex-text]{Arial}
\setmonofont{Arial Narrow}
\usepackage{listings}
\lstset{backgroundcolor={\color{shadebox}},
basicstyle={\small},
breaklines={true},
frame=single,
frameround=tttt,
framerule={.5pt},
morekeywords={GET,POST,PUT,DELETE},
numbers=left,
numberstyle={\tiny},
otherkeywords={GET,POST,PUT,DELETE,HTTP/1.1, 200,OK,appication/xml},
stringstyle={\ttfamily},
tabsize=4}
\usepackage[letterpaper]{geometry}
\geometry{verbose,tmargin=1.8in,bmargin=1.25in,lmargin=1in,rmargin=1in}
\usepackage{fancyhdr}
\pagestyle{fancy}
\setcounter{secnumdepth}{5}
\setcounter{tocdepth}{5}
\setlength{\parskip}{\bigskipamount}
\setlength{\parindent}{0pt}
\usepackage{color}
\usepackage{array}
\usepackage{verbatim}
\usepackage{longtable}

\makeatletter

%%%%%%%%%%%%%%%%%%%%%%%%%%%%%% LyX specific LaTeX commands.
%% Because html converters don't know tabularnewline
\providecommand{\tabularnewline}{\\}

%%%%%%%%%%%%%%%%%%%%%%%%%%%%%% Textclass specific LaTeX commands.

\numberwithin{equation}{section}
\numberwithin{figure}{section}
\usepackage{enumitem}		% customizable list environments
\newlength{\lyxlabelwidth}      % auxiliary length 
\numberwithin{table}{section}
\newenvironment{lyxcode}
{\par\begin{list}{}{
\setlength{\rightmargin}{\leftmargin}
\setlength{\listparindent}{0pt}% needed for AMS classes
\raggedright
\setlength{\itemsep}{0pt}
\setlength{\parsep}{0pt}
\normalfont\ttfamily}%
 \item[]}
{\end{list}}

%%%%%%%%%%%%%%%%%%%%%%%%%%%%%% User specified LaTeX commands.
\usepackage[usenames,dvipsnames,svgnames,table]{xcolor}
\usepackage{fancyhdr}
\usepackage{eso-pic}
\usepackage[stamp]{draftwatermark}
\usepackage{colortbl}

\SetWatermarkText{DPS Internal}
\SetWatermarkAngle{45}
\SetWatermarkLightness{0.9}
\SetWatermarkScale{.8}

\date{}
\lhead{\small{www.dedoimedo.com}}
\rhead{\small{some text}} 
\pagestyle{fancy}

\setlength{\headheight}{0.6in}
\setlength{\headwidth}{\textwidth}
\fancyhead[L]{}% empty left
\fancyhead[R]{ % right
   \includegraphics[height=0.75in]{../common/att_logo.png}
}
\pagestyle{fancy}

\definecolor{blue}{RGB}{79,129,189}
\definecolor{shadebox}{RGB}{255,255,204}

\cfoot{\scriptsize © 2013 AT\&T Intellectual Property. All rights reserved. AT\&T and AT\&T logo are trademarks of AT\&T Intellectual Property.}

\newcommand{\attcategory}[1]{\textbf{\Large Category: \textcolor{blue}{#1}}}
\newcommand{\attservice}[1]{\textbf{\Large Service:  \textcolor{blue}{#1}}}
\newcommand{\attdocumentnumber}[1]{\textbf{\small Document Number: #1}}
\newcommand{\attrevision}[1]{\textbf{\small Revision: #1}}
\newcommand{\attrevisiondate}[1]{\textbf{\small Revision Date: #1}}
\newcommand{\attauthor}[1]{\textbf{\small Author: #1}}


\let\oldtabular=\tabular
\def\tabular{\footnotesize\oldtabular}

\makeatother

\usepackage{xunicode}
\usepackage{polyglossia}
\setdefaultlanguage{english}
\begin{document}

\subsubsection{Functional Behavior}

\begin{comment}
Describe the functional behavior provided by this operation. Also
add here any special considerations, assumptions, and dependencies
for this particular operation
\end{comment}


Add a list of file into an existing folder. Adding the same file twice
in a folder is allowed. A folder “Antarctica” contains File1.txt

If a user selects 3 files, File1.txt, File2.txt, File3.txt and adds
them to the folder “Antarctica”, folder “Antarctica” will contain,
the following: File1.txt, File1.txt, File2.txt and File3.txt.


\subsubsection{Call flow}

\begin{comment}
Insert below a sequence diagram (ping pong) diagram in context to
a complete call flow or transaction with numbered arrows and some
description explaining the sequence as appropriate.
\end{comment}



\subsubsection{Version Impact Summary}

\begin{comment}
Summarize the different versions of this operation that are in production
or deleted by the current release of this service. 
\end{comment}


\begin{longtable}{|>{\raggedright}p{0.15\textwidth}|>{\raggedright}p{0.16\textwidth}|>{\raggedright}p{0.55\textwidth}|}
\hline
\hline 
\textbf{\footnotesize{Service Version}} & \textbf{\footnotesize{Major or Minor Impact}} & \textbf{\footnotesize{Changes Introduced by this Version}}\tabularnewline
\hline 
\hline
\endfirsthead
\hline
\hline 
\textbf{\footnotesize{Service Version}} & \textbf{\footnotesize{Major or Minor Impact}} & \textbf{\footnotesize{Changes Introduced by this Version}}\tabularnewline
\hline 
\hline
\endhead
\hline 
{\footnotesize{1}} & {\footnotesize{Major}} & {\footnotesize{Initial Release}}\tabularnewline
\hline 
\end{longtable}


\subsubsection{Authentication and Authorization}

\begin{comment}
Describe OAuth model applicable to this operation.
\end{comment}


\begin{longtable}{|>{\raggedright}p{0.18\textwidth}|>{\raggedright}p{0.1\textwidth}|>{\raggedright}p{0.14\textwidth}|>{\raggedright}p{0.44\textwidth}|}
\hline
\hline 
\textbf{\footnotesize{Authorization Model}} & \textbf{\footnotesize{Subscriber Authorization required?}} & \textbf{\footnotesize{OAuth Scope Value}} & \textbf{\footnotesize{Brief Description}}\tabularnewline
\hline 
\hline
\endfirsthead
\hline
\hline 
\textbf{\footnotesize{Authorization Model}} & \textbf{\footnotesize{Subscriber Authorization required?}} & \textbf{\footnotesize{OAuth Scope Value}} & \textbf{\footnotesize{Brief Description}}\tabularnewline
\hline 
\hline
\endhead
\hline 
{\footnotesize{OAuth }}{\footnotesize \par}

{\footnotesize{authorization\_code}} & {\footnotesize{Yes}} & {\footnotesize{LOCKER}} & \begin{enumerate}
\item {\footnotesize{Redirect user to AT\&T authorization page to capture
User consent}}{\footnotesize \par}
\item {\footnotesize{Obtain Token obtained using authorization code issued
by the gateway alonf the client key and secret by making an Oauth
token call with GrantType=authorization\_code}}\end{enumerate}
\tabularnewline
\hline 
\end{longtable}


\subsubsection{Representation Formats}

\begin{comment}
In the table below, describe supported representation formats for
the request and response body. Where different versions support different
formats, provide separate tables and indicate which versions apply
to each table. 
\end{comment}


\begin{longtable}{|>{\raggedright}p{0.16\textwidth}|>{\raggedright}p{0.7\textwidth}|}
\hline
\hline 
\textbf{\footnotesize{Direction}} & \textbf{\footnotesize{Supported Representation Formats}}\tabularnewline
\hline 
\hline
\endfirsthead
\hline
\hline 
\textbf{\footnotesize{Direction}} & \textbf{\footnotesize{Supported Representation Formats}}\tabularnewline
\hline 
\hline
\endhead
\hline 
{\footnotesize{Request}} & {\footnotesize{XML, JSON}}\tabularnewline
\hline 
{\footnotesize{Response}} & {\footnotesize{XML, JSON}}\tabularnewline
\hline 
\end{longtable}






\subsubsection{Input Parameters}

\begin{comment}
In the table below, describe input parameters. Where different versions
support different input parameters, provide separate tables and indicate
which versions apply to each table. The allowed values for the Location
in Request column are: Header, Resource URI, Body or Query Parameter.
\end{comment}


{\footnotesize{\begin{longtable}{|>{\raggedright}p{0.23\textwidth}|>{\raggedright}p{0.08\textwidth}|>{\raggedright}p{0.05\textwidth}|>{\raggedright}p{0.41\textwidth}|>{\raggedright}p{0.1\textwidth}|}
\hline
\hline 
\textbf{\footnotesize{Parameter }} & \textbf{\footnotesize{Data Type}} & \textbf{\footnotesize{Req?}} & \textbf{\footnotesize{Brief description}} & \textbf{\footnotesize{Location}}\tabularnewline
\hline 
\hline
\endfirsthead
\hline
\hline 
\textbf{\footnotesize{Parameter }} & \textbf{\footnotesize{Data Type}} & \textbf{\footnotesize{Req?}} & \textbf{\footnotesize{Brief description}} & \textbf{\footnotesize{Location}}\tabularnewline
\hline 
\hline
\endhead
\hline 
{\footnotesize{accept}} & {\footnotesize{String}} & {\footnotesize{No}} & {\footnotesize{Specifies the format of the body of the response. Valid
values are: }}{\footnotesize \par}
\begin{itemize}
\item {\footnotesize{application/json}}{\footnotesize \par}
\item {\footnotesize{application/xml}}{\footnotesize \par}
\end{itemize}
{\footnotesize{The default value is application/json. Per rfc2616:
\textquotedbl{}If no Accept header field is present, then it is assumed
that the client accepts all media types.\textquotedbl{} By default
our services return application/json.}}{\footnotesize \par}

{\footnotesize{Normal Accept header processing rules shall be followed
according to rfc2616.}}{\footnotesize \par}

\emph{\footnotesize{Note}}{\footnotesize{: If there is no entity body
in a normal successful response, this parameter can still be specified
to determine the format in the case of an error response message.}} & {\footnotesize{Header}}\tabularnewline
\hline 
\end{longtable}
}}{\footnotesize \par}


\paragraph{Request – Example (Add files to a folder in XML)\protect \\
}



This demonstrates how to add a file to a folder, and specifying the
request to be in XML.

\begin{lstlisting}[backgroundcolor={\color{shadebox}},basicstyle={\footnotesize\ttfamily},frame=single,framerule={.5pt}]
POST /locker/v1/folders/3433435/files HTTP/1.1 
Host: api.att.com
content-type: application/xml
authorization: Bearer abcdef12345678
accept: application/xml

<?xml version="1.0" encoding="UTF-8" ?>
<fileIds>
    <id>112</id>
    <id>125</id>
    <index>1</index>  	
</fileIds>
\end{lstlisting}



\paragraph{Request – Example (Add files to a folder in JSON)\protect \\
}

This demonstrates how to add a file to a folder, and specifying the
request to be in JSON.

\begin{lstlisting}[backgroundcolor={\color{shadebox}},basicstyle={\footnotesize\ttfamily},frame=single,framerule={.5pt}]
POST /locker/v1/folders/3433435/files HTTP/1.1 
Host: api.att.com
content-type: application/json
authorization: Bearer abcdef12345678
accept: application/json

{
  "fileIds": {
    "id": [112,125],
    "index": 1
  }
}
\end{lstlisting}



\subsubsection{Output Parameters}

\begin{comment}
In the table below, describe output parameters. Where different versions
support different output parameters, provide separate tables and indicate
which versions apply to each table. Note: ‘consent’ refers to end
user consent which may need to be obtained before data is returned,
\end{comment}


{\footnotesize{{\footnotesize{}}%
\begin{longtable}{|>{\raggedright}p{0.23\textwidth}|>{\raggedright}p{0.08\textwidth}|>{\raggedright}p{0.05\textwidth}|>{\raggedright}p{0.41\textwidth}|>{\raggedright}p{0.1\textwidth}|}
\hline
\hline 
\textbf{\footnotesize{Parameter }} & \textbf{\footnotesize{Data Type}} & \textbf{\footnotesize{Req?}} & \textbf{\footnotesize{Brief description}} & \textbf{\footnotesize{Location}}\tabularnewline
\hline 
\hline
\endfirsthead
\hline
\hline 
\textbf{\footnotesize{Parameter }} & \textbf{\footnotesize{Data Type}} & \textbf{\footnotesize{Req?}} & \textbf{\footnotesize{Brief description}} & \textbf{\footnotesize{Location}}\tabularnewline
\hline 
\hline
\endhead
\hline 
{\footnotesize{content-type}} & {\footnotesize{string}} & {\footnotesize{yes}} & {\footnotesize{Specifies the type of content of the body of the entity.
Must be set to one of the following values:}}{\footnotesize \par}
\begin{itemize}
\item {\footnotesize{application/json}}{\footnotesize \par}
\item {\footnotesize{aplication/xml}}{\footnotesize \par}
\end{itemize}
{\footnotesize{Note: If there is no entity body in a normal successful
response, this parameter is still needed to specify the format in
the case of an error response message.}} & {\footnotesize{Header}}\tabularnewline
\hline 
\end{longtable}{\footnotesize \par}

{\footnotesize{}}
}}{\footnotesize \par}


\paragraph{Response – Example (Response to adding files to a folder)\protect \\
}

This shows the response to a an add file to folder request.

\texttt{}
\begin{lstlisting}[basicstyle={\footnotesize\ttfamily},tabsize=4]
HTTP/1.1 204 No Content
Date: Thu, 04 Jun 2010 02:51:59 GMT
\end{lstlisting}



\subsubsection{HTTP Response Codes}

\begin{comment}
List expected response codes (indicate any variation that exists across
different versions):
\end{comment}


\begin{longtable}{|>{\raggedright}p{0.15\textwidth}|>{\raggedright}p{0.15\textwidth}|>{\raggedright}p{0.56\textwidth}|}
\hline
\hline 
\textbf{\footnotesize{Code}} & \textbf{\footnotesize{Reason Phrase}} & \textbf{\footnotesize{Description}}\tabularnewline
\hline
\endfirsthead
\hline
\hline 
\textbf{\footnotesize{Code}} & \textbf{\footnotesize{Reason Phrase}} & \textbf{\footnotesize{Description}}\tabularnewline
\hline
\endhead
\hline 
{\footnotesize{204}} & {\footnotesize{No Content}} & {\footnotesize{Successful response and the response does not include
an entity.}}\tabularnewline
\hline 
{\footnotesize{400}} & {\footnotesize{Bad Request}} & {\footnotesize{Many possible reasons not specified by the other codes.}}\tabularnewline
\hline 
{\footnotesize{401}} & {\footnotesize{Authentication Error}} & {\footnotesize{Authentication failed or was not provided.}}\tabularnewline
\hline 
{\footnotesize{403}} & {\footnotesize{Forbidden}} & {\footnotesize{Access permission error.}}\tabularnewline
\hline 
{\footnotesize{404}} & {\footnotesize{Not Found}} & {\footnotesize{The server has not found anything matching the Request-URI.
No indication is given of whether the condition is temporary or permanent.}}\tabularnewline
\hline 
{\footnotesize{405}} & {\footnotesize{Method Not Allowed}} & {\footnotesize{A request was made of a resource using a request method
not supported by that resource (e.g., using PUT on a REST resource
that only supports POST).}}\tabularnewline
\hline 
{\footnotesize{408}} & {\footnotesize{Request Timeout}} & {\footnotesize{The client did not produce a request within the time
that the server was prepared to wait. The client MAY repeat the request
without modifications at any later time. }}\tabularnewline
\hline 
{\footnotesize{411}} & {\footnotesize{Length Required}} & {\footnotesize{The Content-Length header was not specified.}}\tabularnewline
\hline 
{\footnotesize{413}} & {\footnotesize{Request Entity Too Large}} & {\footnotesize{The size of the request body exceed the maximum size
permitted.}}\tabularnewline
\hline 
{\footnotesize{415}} & {\footnotesize{Unsupported Media Type}} & {\footnotesize{The request is in a format not supported by the requested
resource for the requested method.}}\tabularnewline
\hline 
{\footnotesize{500}} & {\footnotesize{Internal Server Error}} & {\footnotesize{The server encountered an internal error or timed out;
please retry.}}\tabularnewline
\hline 
{\footnotesize{503}} & {\footnotesize{Service Unavailable}} & {\footnotesize{The server is currently unable to receive requests;
please retry.}}\tabularnewline
\hline 
\end{longtable}




\subsubsection{Service Exceptions}

\begin{comment}
List the service exceptions generated by the operation (indicate any
variation that exists across different versions):
\end{comment}


\begin{longtable}{|>{\raggedright}p{0.1\textwidth}|>{\raggedright}p{0.33\textwidth}|>{\raggedright}p{0.33\textwidth}|>{\raggedright}p{0.1\textwidth}|}
\hline
\hline 
\textbf{\footnotesize{MessageId}} & \textbf{\footnotesize{Text}} & \textbf{\footnotesize{Variables}} & \textbf{\footnotesize{Parent HTTP Code}}\tabularnewline
\hline 
\hline
\endfirsthead
\hline
\hline 
\textbf{\footnotesize{MessageId}} & \textbf{\footnotesize{Text}} & \textbf{\footnotesize{Variables}} & \textbf{\footnotesize{Parent HTTP Code}}\tabularnewline
\hline 
\hline
\endhead
\hline 
{\footnotesize{SVC0001}} & {\footnotesize{A service error has occurred. Error code is <Error
Explanation>}} & {\footnotesize{Error Explanation : <content here>}} & {\footnotesize{400}}\tabularnewline
\hline 
{\footnotesize{SVC0002}} & {\footnotesize{Invalid input value for Message part <part name>}} & {\footnotesize{part name : name of the input parameter that resulted
in the error}} & {\footnotesize{400}}\tabularnewline
\hline 
\end{longtable}




\subsubsection{Policy Exceptions}

\begin{comment}
List the policy exceptions generated by the operation (indicate any
variation that exists across different versions):
\end{comment}


\begin{longtable}{|>{\raggedright}p{0.1\textwidth}|>{\raggedright}p{0.33\textwidth}|>{\raggedright}p{0.33\textwidth}|>{\raggedright}p{0.1\textwidth}|}
\hline
\hline 
\textbf{\footnotesize{MessageId}} & \textbf{\footnotesize{Text}} & \textbf{\footnotesize{Variables}} & \textbf{\footnotesize{Parent HTTP Code}}\tabularnewline
\hline 
\hline
\endfirsthead
\hline
\hline 
\textbf{\footnotesize{MessageId}} & \textbf{\footnotesize{Text}} & \textbf{\footnotesize{Variables}} & \textbf{\footnotesize{Parent HTTP Code}}\tabularnewline
\hline 
\hline
\endhead
\hline 
{\footnotesize{POL0001}} & {\footnotesize{A policy error occurred. For example, rate limit error,
authentication and authorization error.}} & {\footnotesize{N/A}} & {\footnotesize{401,403}}\tabularnewline
\hline 
{\footnotesize{POL1009}} & {\footnotesize{User has not been provisioned for \%1}} & {\footnotesize{System that hasn’t been provisioned}} & {\footnotesize{403}}\tabularnewline
\hline 
\end{longtable}

\end{document}
