\begin{longtable}{|>{\raggedright}p{0.23\textwidth}|>{\raggedright}p{0.08\textwidth}|>{\raggedright}p{0.1\textwidth}|>{\raggedright}p{0.36\textwidth}|>{\raggedright}p{0.1\textwidth}|}
\hline
\hline 
\textbf{\footnotesize{Parameter }} & \textbf{\footnotesize{Data Type}} & \textbf{\footnotesize{Req?}} & \textbf{\footnotesize{Brief description}} & \textbf{\footnotesize{Location}}\tabularnewline
\hline 
\hline
\endfirsthead
\hline
\hline 
\textbf{\footnotesize{Parameter }} & \textbf{\footnotesize{Data Type}} & \textbf{\footnotesize{Req?}} & \textbf{\footnotesize{Brief description}} & \textbf{\footnotesize{Location}}\tabularnewline
\hline 
\hline
\endhead
\hline 
{\footnotesize{accept}} & {\footnotesize{String}} & {\footnotesize{No}} & {\footnotesize{Specifies the format of the body of the response. The
acceptable values for this parameter are:}}{\footnotesize \par}
\begin{itemize}
\item {\footnotesize{application/json}}{\footnotesize \par}
\item {\footnotesize{application/xml}}{\footnotesize \par}
\item {\footnotesize{application/x-www-form-urlencoded}}{\footnotesize \par}
\end{itemize}
{\footnotesize{The default value is application/json. }}{\footnotesize \par}

{\footnotesize{Per rfc2616: \textquotedbl{}If no Accept header field
is present, then it is assumed that the client accepts all media types.\textquotedbl{}
By default our services return application/json.}}{\footnotesize \par}

{\footnotesize{The normal Accept header processing rules shall be
followed according to rfc2616.}}{\footnotesize \par}

\emph{\footnotesize{Note}}{\footnotesize{: If there is no entity body
in a normal successful response, this parameter is still needed to
specify the format in the case of an error response message.}} & {\footnotesize{Header}}\tabularnewline
\hline 
{\footnotesize{authorization}} & {\footnotesize{String}} & {\footnotesize{Yes}} & {\footnotesize{Specifies the authorization type and token. The acceptable
format for this parameter is the phrase \textquotedbl{}Bearer OAuth
Token\textquotedbl{} followed by a space ( ) and an OAuth access token.
If this parameter value is missing from the header, then the API Gateway
returns a message with an HTTP status code of 400 Invalid Request.
If the OAuth access token is not valid, then the API Gateway returns
an HTTP status code of 401 Unauthorized with a WWW-Authenticate HTTP
header.}} & {\footnotesize{Header}}\tabularnewline
\hline 
{\footnotesize{content-length}} & {\footnotesize{Integer}} & {\footnotesize{No}} & {\footnotesize{Specifies the length of the content in octets. This
parameter is only required for a non-streaming request.}} & {\footnotesize{Header}}\tabularnewline
\hline 
{\footnotesize{content-type}} & {\footnotesize{String}} & {\footnotesize{Yes}} & {\footnotesize{Specifies the type of content of the body. The acceptable
values for this parameter are:}}{\footnotesize \par}
\begin{itemize}
\item {\footnotesize{application/xml}}{\footnotesize \par}
\item {\footnotesize{application/json}}\end{itemize}
 & {\footnotesize{Header}}\tabularnewline
\hline 
{\footnotesize{transfer-encoding}} & {\footnotesize{String}} & {\footnotesize{Conditional}} & {\footnotesize{Specifies the encodingof the message. The only acceptable
values for this parameter is:}}{\footnotesize \par}
\begin{itemize}
\item {\footnotesize{chunked}}{\footnotesize \par}
\end{itemize}
{\footnotesize{This parameter is only required for a streaming request.}} & {\footnotesize{Header}}\tabularnewline
\hline 
{\footnotesize{x-invitation}} & {\footnotesize{String}} & {\footnotesize{Yes}} & {\footnotesize{Specifies the answer to the invitation. The acceptable
values for this parameter are:}}{\footnotesize \par}
\begin{itemize}
\item {\footnotesize{accept : The participant accepts the chat invitation.}}{\footnotesize \par}
\item {\footnotesize{decline : The participant rejects the chat invitation.}}{\footnotesize \par}
\item {\footnotesize{join : Th participant joins ongoing group chat conversation.}}\end{itemize}
 & {\footnotesize{Header}}\tabularnewline
\hline 
\end{longtable}



{\footnotesize{}}
