\begin{comment}
Describe OAuth model applicable to this operation.
\end{comment}


\begin{longtable}{|>{\raggedright}p{0.18\textwidth}|>{\raggedright}p{0.1\textwidth}|>{\raggedright}p{0.14\textwidth}|>{\raggedright}p{0.44\textwidth}|}
\hline
\hline 
\textbf{\footnotesize{Authorization Model}} & \textbf{\footnotesize{Subscriber Authorization required?}} & \textbf{\footnotesize{OAuth Scope Value}} & \textbf{\footnotesize{Brief Description}}\tabularnewline
\hline 
\hline
\endfirsthead
\hline
\hline 
\textbf{\footnotesize{Authorization Model}} & \textbf{\footnotesize{Subscriber Authorization required?}} & \textbf{\footnotesize{OAuth Scope Value}} & \textbf{\footnotesize{Brief Description}}\tabularnewline
\hline 
\hline
\endhead
\hline 
{\footnotesize{authorization\_code}} & {\footnotesize{Yes}} & {\footnotesize{RTC}} & \begin{enumerate}
\item {\footnotesize{Obtain the authorization code by passing App Key and
App Secret along with scope. This redirects the user to AT\&T authorization
page to capture user consent.}}{\footnotesize \par}
\item {\footnotesize{Obtain OAuth access token using the OAuth authorization
code along with the App Key and App Secret by making a Get Access
Token request with grant\_type=authorization\_code.}}{\footnotesize \par}

\begin{itemize}
\item {\footnotesize{Supports ICMN case}}\end{itemize}
\end{enumerate}
\tabularnewline
\hline 
{\footnotesize{client\_credentials}} & {\footnotesize{No}} & {\footnotesize{RTC}} & \begin{enumerate}
\item {\footnotesize{Obtain OAuth access token with the App Key and App
Secret along with the scope by making a Get Access Token request with
grant\_type=client\_credentials}}{\footnotesize \par}

\begin{itemize}
\item {\footnotesize{Supports VTN case }}{\footnotesize \par}
\item {\footnotesize{Supports no-TN case }}\end{itemize}
\end{enumerate}
\tabularnewline
\hline 
\end{longtable}
