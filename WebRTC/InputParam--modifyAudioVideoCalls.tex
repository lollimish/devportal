{\footnotesize{}}%
\begin{longtable}{|>{\raggedright}p{0.18\textwidth}|>{\raggedright}p{0.08\textwidth}|>{\raggedright}p{0.09\textwidth}|>{\raggedright}p{0.4\textwidth}|>{\raggedright}p{0.11\textwidth}|}
\hline
\hline 
\textbf{\footnotesize{Parameter }} & \textbf{\footnotesize{Data Type}} & \textbf{\footnotesize{Required?}} & \textbf{\footnotesize{Brief description}} & \textbf{\footnotesize{Location}}\tabularnewline
\hline 
\hline
\endfirsthead
\hline
\hline 
\textbf{\footnotesize{Parameter }} & \textbf{\footnotesize{Data Type}} & \textbf{\footnotesize{Required?}} & \textbf{\footnotesize{Brief description}} & \textbf{\footnotesize{Location}}\tabularnewline
\hline 
\hline
\endhead
\hline 
{\footnotesize{accept}} & {\footnotesize{String}} & {\footnotesize{No}} & {\footnotesize{Specifies the format of the body of the response. The
acceptable values for this parameter are:}}{\footnotesize \par}
\begin{itemize}
\item {\footnotesize{application/json}}{\footnotesize \par}
\item {\footnotesize{application/xml}}{\footnotesize \par}
\item {\footnotesize{application/x-www-form-urlencoded}}{\footnotesize \par}
\end{itemize}
{\footnotesize{The default value is application/json. }}{\footnotesize \par}

{\footnotesize{Per rfc2616: \textquotedbl{}If no Accept header field
is present, then it is assumed that the client accepts all media types.\textquotedbl{}
By default our services return application/json.}}{\footnotesize \par}

{\footnotesize{The normal Accept header processing rules shall be
followed according to rfc2616.}}{\footnotesize \par}

\emph{\footnotesize{Note}}{\footnotesize{: If there is no entity body
in a normal successful response, this parameter is still needed to
specify the format in the case of an error response message.}} & {\footnotesize{Header}}\tabularnewline
\hline 
{\footnotesize{authorization}} & {\footnotesize{String}} & {\footnotesize{Yes}} & {\footnotesize{Specifies the authorization type and token. The acceptable
format for this parameter is the phrase \textquotedbl{}Bearer OAuth
Token\textquotedbl{} followed by a space ( ) and an OAuth access token.
If this parameter value is missing from the header, then the API Gateway
returns a message with an HTTP status code of 400 Invalid Request.
If the OAuth access token is not valid, then the API Gateway returns
an HTTP status code of 401 Unauthorized with a WWW-Authenticate HTTP
header.}} & {\footnotesize{Header}}\tabularnewline
\hline 
{\footnotesize{x-calls-action}} & {\footnotesize{String}} & {\footnotesize{Yes}} & {\footnotesize{Specifies the action for a conference session . The
acceptable values for this parameter are:}}{\footnotesize \par}
\begin{itemize}
\item {\footnotesize{call-answer : The client accepts the call invitation.}}{\footnotesize \par}
\item {\footnotesize{initiate-call-mod : The client request for change in
media attributes, for example remove a video stream from an ongoing
call session.}}{\footnotesize \par}
\item {\footnotesize{initiate-call-hold : The Client request to put the
media path in a hold state.}}{\footnotesize \par}
\item {\footnotesize{initiate-call-transfer : The client put the call in
a hold state, calls another number, and bridges the two parties. }}{\footnotesize \par}
\item {\footnotesize{initiate-call-resume : The Client request to resume
the media path which was in a hold state.}}{\footnotesize \par}
\item {\footnotesize{initiate-call-move : The Client requests to move the
call to different registered active endpoint. }}{\footnotesize \par}
\item {\footnotesize{accept-call-mod : The client accepts the changes in
media attributes, for example remove a video stream from an ongoing
conference session.}}{\footnotesize \par}
\item {\footnotesize{rejectCancel-call-mod : The proposal from the remote
client to reject a media modification in an AudioVideo mod-received
event or the Initiate WebRTC Session request from the local client
to cancel a media modification.}}\end{itemize}
 & {\footnotesize{Header}}\tabularnewline
\hline 
{\footnotesize{x-transferTargetCallId}} & {\footnotesize{String}} & {\footnotesize{Yes}} & {\footnotesize{Specifies the call identifier of the transfer target.}} & {\footnotesize{Header}}\tabularnewline
\hline 
{\footnotesize{x-modId}} & {\footnotesize{String}} & {\footnotesize{Conditional}} & {\footnotesize{Specifies the modification identifier for media changes.}}{\footnotesize \par}

{\footnotesize{This parameter is required if the x-calls-action parameter
is set to accept-call-mod or rejectCancel-call-mod.}} & {\footnotesize{Header}}\tabularnewline
\hline 
{\footnotesize{x-reject-reason}} & {\footnotesize{String}} & {\footnotesize{No}} & {\footnotesize{Specifies the reason for rejecting the session media
changes.}}{\footnotesize \par}

{\footnotesize{This parameter is only applicable if the x-calls-action
parameter is set to rejectCancel-call-mod.}} & {\footnotesize{Header}}\tabularnewline
\hline 
{\footnotesize{callsMediaModifications}} & \multirow{1}{0.08\textwidth}{{\footnotesize{callsMediaModifications Object}}} & {\footnotesize{Yes}} & {\footnotesize{Contains the call media information.}} & {\footnotesize{Body}}\tabularnewline
\hline 
\end{longtable}{\footnotesize \par}

{\footnotesize{}}{\footnotesize \par}

\textbf{\footnotesize{Structure of callsMediaModifications Object}}{\footnotesize \par}

{\footnotesize{}}%
\begin{tabular}{|c|c|c|>{\centering}p{0.4\textwidth}|}
\hline 
\textbf{\footnotesize{Parameter}} & \textbf{\footnotesize{Data Type}} & \textbf{\footnotesize{Required?}} & \textbf{\footnotesize{Brief description}}\tabularnewline
\hline 
\hline 
{\footnotesize{sdp}} & {\footnotesize{Sdp Object}} & {\footnotesize{Yes}} & {\small{Contains an SDP offer for the audio video session.}}\tabularnewline
\hline 
\end{tabular}{\footnotesize \par}

\textbf{\footnotesize{Structure of sdp Object}}{\footnotesize \par}

{\footnotesize{}}%
\begin{longtable}{|c||c||c||>{\raggedright}p{0.4\textwidth}|}
\hline 
\textbf{\footnotesize{Parameter}} & \textbf{\footnotesize{Data Type}} & \textbf{\footnotesize{Required?}} & \textbf{\footnotesize{Brief description}}\tabularnewline
\hline 
\hline 
{\footnotesize{m}} & {\footnotesize{String}} & {\footnotesize{Yes}} & {\small{Specifies the media description in the following format.}}{\small \par}

{\small{m=<media> <port> <proto> <fmt>}}{\small \par}

{\small{The <media> field is the media type. }}{\footnotesize{The
only acceptable values for this field is:}}{\footnotesize \par}
\begin{itemize}
\item {\small{audio}}{\small \par}
\end{itemize}
{\small{The <port> field is the transport port to which the media
stream is sent.}}{\small \par}

{\small{<proto> is the transport protocol. }}{\footnotesize{The only
acceptable values for this field is:}}{\footnotesize \par}
\begin{itemize}
\item {\small{TCP/RTMP}}{\small \par}
\end{itemize}
{\small{The <fmt> field is a media format description.}}{\small \par}

{\small{The fourth and any subsequent sub-fields describe the format
of the media. The interpretation of the media format depends on the
value of the <proto> sub-field.}}\tabularnewline
\hline 
\hline 
{\footnotesize{attributes}} & {\footnotesize{List of Strings}} & {\footnotesize{Yes}} & {\small{Contains the attributes for extending SDP. This parameter
may be defined to be used as session-level attributes, media-level
attributes, or both.}}{\small \par}

{\small{The attribute fields may be in one of the following formats.}}{\small \par}

{\small{A property attribute is simply of the form:}}{\small \par}
\begin{itemize}
\item {\small{a=<flag> }}{\small \par}
\item {\small{These are binary attributes, and the presence of the attribute
conveys that the attribute is a property of the session. Example:
a=recvonly}}{\small \par}
\end{itemize}
{\small{A value attribute is of the form:}}{\small \par}
\begin{itemize}
\item {\small{a=<attribute>:<value>}}\end{itemize}
\tabularnewline
\hline 
\end{longtable}
