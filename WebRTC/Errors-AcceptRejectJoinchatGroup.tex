
\subsubsection{Service Exceptions}

\begin{comment}
List the service exceptions generated by the operation (indicate any
variation that exists across different versions):
\end{comment}


\begin{longtable}{|>{\raggedright}p{0.1\textwidth}|>{\raggedright}p{0.33\textwidth}|>{\raggedright}p{0.33\textwidth}|>{\raggedright}p{0.1\textwidth}|}
\hline
\hline 
\textbf{\footnotesize{MessageId}} & \textbf{\footnotesize{Text}} & \textbf{\footnotesize{Variables}} & \textbf{\footnotesize{Parent HTTP Code}}\tabularnewline
\hline 
\hline
\endfirsthead
\hline
\hline 
\textbf{\footnotesize{MessageId}} & \textbf{\footnotesize{Text}} & \textbf{\footnotesize{Variables}} & \textbf{\footnotesize{Parent HTTP Code}}\tabularnewline
\hline 
\hline
\endhead
\hline 
{\footnotesize{SVC0001}} & {\footnotesize{A service error has occurred. Error code is <error\_explanation>}} & {\footnotesize{error\_explanation : <content\_here>}} & {\footnotesize{400}}\tabularnewline
\hline 
{\footnotesize{SVC0002}} & {\footnotesize{Invalid input value for Message part <part\_name>}} & {\footnotesize{part\_name : name of the input parameter that resulted
in the error.}} & {\footnotesize{400}}\tabularnewline
\hline 
{\footnotesize{SVC0003}} & {\footnotesize{Invalid input value for Message part <part\_name>,
valid values are <part\_values>}} & {\footnotesize{part\_name : name of the input parameter that resulted
in the error.}}{\footnotesize \par}

{\footnotesize{part\_value : value of input parameter that was found
to be in error.}} & {\footnotesize{400}}\tabularnewline
\hline 
{\footnotesize{SVC0004}} & {\footnotesize{No valid addresses provided in the Message part <part\_name>}} & {\footnotesize{part\_name : name of the input parameter that resulted
in the error.}} & {\footnotesize{400}}\tabularnewline
\hline 
\end{longtable}


\subsubsection{Policy Exceptions}

\begin{comment}
List the policy exceptions generated by the operation (indicate any
variation that exists across different versions):
\end{comment}


\begin{longtable}{|>{\raggedright}p{0.1\textwidth}|>{\raggedright}p{0.33\textwidth}|>{\raggedright}p{0.33\textwidth}|>{\raggedright}p{0.1\textwidth}|}
\hline
\hline 
\textbf{\footnotesize{MessageId}} & \textbf{\footnotesize{Text}} & \textbf{\footnotesize{Variables}} & \textbf{\footnotesize{Parent HTTP Code}}\tabularnewline
\hline 
\hline
\endfirsthead
\hline
\hline 
\textbf{\footnotesize{MessageId}} & \textbf{\footnotesize{Text}} & \textbf{\footnotesize{Variables}} & \textbf{\footnotesize{Parent HTTP Code}}\tabularnewline
\hline 
\hline
\endhead
\hline 
{\footnotesize{POL0001}} & {\footnotesize{A policy error occurred. For example, rate limit error,
authentication and authorization error.}} & {\footnotesize{N/A}} & {\footnotesize{401, 403}}\tabularnewline
\hline 
{\footnotesize{POL0002}} & {\footnotesize{Privacy verification failed for address <address>,
request is refused}} & {\footnotesize{address : <content\_here>}} & {\footnotesize{403}}\tabularnewline
\hline 
{\footnotesize{POL0003}} & {\footnotesize{Too many addresses specified in Message part}} & {\footnotesize{N/A}} & {\footnotesize{403}}\tabularnewline
\hline 
{\footnotesize{POL1009}} & {\footnotesize{User has not been provisioned for \%1}} & {\footnotesize{1\% : System that has not been provisioned}} & {\footnotesize{403}}\tabularnewline
\hline 
\end{longtable}
